% !Mode:: "TeX:UTF-8"
% +-----------------------------------------------------------------------------
% | File: resume-zh
% | Author: huxuan
% | E-mail: i(at)huxuan.org
% | Created: 2012-12-18
% | Last modified: 2013-03-16
% | Description:
% |     A Chinese Resume Example in LaTeX based on resumecls
% |
% | Copyrgiht (c) 2012-2013 by huxuan. All rights reserved.
% +-----------------------------------------------------------------------------

\documentclass[fontset=fandol,zihao=false,scheme=chinese,heading=true,10pt]{resumecls}
\usepackage[default]{sourcesanspro}
% \usepackage{fontawesome}
\usepackage{sourcecodepro}
% \ctexset{today=small}
\providecommand{\tightlist}{%
  \setlength{\itemsep}{0pt}\setlength{\parskip}{0pt}}
  
\name{黄湘云}
\organization{学校:中国矿业大学(北京)}
\address{住址:北京市海淀区学院路丁11号 \quad 邮编:100083}
\mobile{电话:188 1097 2907}
\mail{邮件:xiangyunfaith@outlook.com}
\homepage{https://xiangyun.netlify.com/}
% \leftfooter{最后更新: \today}
% \rightfooter{\url{http://example.com/resume-zh.pdf}}

\begin{document}

\begin{table}

\maketitle

%%%%%%%%%%%%%%%%%%%%%%%%%%%%%%%%%%%%%%%%%%%%%%%%%%%%%%%%%%%%%%%%%%%%%%%%%%%%%%%
\heading{教育经历}
\entry{2em}{Xrp{8em}}{%
    \heiti{中国矿业大学(北京)} & 北京市 & 2015.09--2018.12 \\
}
\entry{4em}{lXX}{%
    理学硕士学位 \quad 理学院 \quad 统计学 \\
	毕业论文:《空间广义线性混合效应模型及其应用》  \\
	研究方向:数据分析与统计计算 
}

\entry{2em}{Xrp{8em}}{%
    \heiti{中国矿业大学(北京)} & 北京市 & 2011.09--2015.06 \\
}
\entry{4em}{lXX}{%
    理学学士学位  \quad 理学院 \quad 数学与应用数学 \\
	% 毕业论文:《统计方法在区域经济社会发展综合评价中的应用》 &  &  \\
}

%%%%%%%%%%%%%%%%%%%%%%%%%%%%%%%%%%%%%%%%%%%%%%%%%%%%%%%%%%%%%%%%%%%%%%%%%%%%%%
\heading{实习经历}
\entry{2em}{Xp{8em}}{%
    \heiti{北京一流科技有限公司} & 机器学习实习生 \\
}
\entry{4em}{X}{工作的具体内容:}
\entry{6em}{X}{%
    将时域和频域方法与深度学习方法应用于预测电力消费,涉及数据清洗,可视化,指数平滑,
ARMA 和GARCH 模型,贝叶斯时间序列分析工具prophet(包含频谱分析),尖点检测,长短期记忆网络LSTM
及Keras 工具 \\
}

\entry{2em}{Xp{8em}}{%
    \heiti{新浪公司} & 数据分析师实习生 \\
}
\entry{4em}{X}{工作的具体内容:}
\entry{6em}{X}{%
    新浪新闻客户端日志分析,主要使用SQL 从ClickHouse 数据仓库提取数据,R 语言分析和可视化,
并完成日报,用数据分析协助其他部门决策,如服务器资源调度等。\\
}
%%%%%%%%%%%%%%%%%%%%%%%%%%%%%%%%%%%%%%%%%%%%%%%%%%%%%%%%%%%%%%%%%%%%%%%%%%%%%%%
\heading{校园及社会实践}
\entry{2em}{Xp{8em}}{%
    院团委宣传部干事 & 2011.10--2012.06 \\
    校红会学生分会组织部长 & 2012.06--2013.06 \\
	校红十字学生分会组织的爱心支教活动 & 2012.09--2012.10 \\
	大学生科研创新训练项目组长  & 2012.09--2014.12\\
	CDA数据分析师培训      & 2016.04--2016.06 \\
	北京大学高维统计短期课程(高维统计与网络分析) & 2016.06\\
	国家自然科学基金项目(编号:11671398)项目组成员& 2017.01--2020.12\\
	研究生助教(数学建模、数值分析、运筹学等课程) & 2015.09--2016.12
}

%%%%%%%%%%%%%%%%%%%%%%%%%%%%%%%%%%%%%%%%%%%%%%%%%%%%%%%%%%%%%%%%%%%%%%%%%%%%%%%
\heading{资格证书及获奖情况}
\entry{2em}{Xr}{%
    全国大学英语四级(448)和六级(462) & 2011.06--2014.06 \\
    2013年高教社杯全国大学生数学建模竞赛(北京市二等奖) & 2013 \\
	2012年本科生优秀学生奖学金(三等奖学金) & 2012 \\
	2013年本科生优秀学生奖学金(二等奖学金) & 2013 \\
	% 第29届全国部分地区物理竞赛(三等奖)& 2012\\
	% 校科技文化节科普创作大赛(优秀奖) & 2013 \\
	% 校物理实验竞赛(一等奖) & 2013\\
	2015-2016年度研究生优秀学生(一等奖学金) & 2016 \\
	2016-2017年度研究生优秀学生荣誉称号 & 2017\\
}
\end{table}

\begin{table}
%%%%%%%%%%%%%%%%%%%%%%%%%%%%%%%%%%%%%%%%%%%%%%%%%%%%%%%%%%%%%%%%%%%%%%%%%%%%%%
\heading{数据可视化}
\entry{2em}{Xp{8em}}{%
    {\heiti 空间数据可视化} \\
	案例1:从中国地震台网获取2012年4月至2017年8月5级及以上的地震数据\\
	案例2:从美国地质调查局获取1973年至2010年世界各地6级及以上地震数据集\\
	主要工作:基于 ggplot2和animotion等R包,制作动态、平面和立体时空数据可视化\\
	{\heiti 网络数据可视化} \\
	分析目标:可视化R语言社区中开发者相互合作的关系网络、挖掘出对社区贡献大的组织、个人\\
	主要工作:抓取CRAN官方11000+ R包信息、RStudio官网下载日志,正则表达式提取开发者、贡献者字段,清洗后,建立有向图网络,使用ggplot2等可视化包,呈现贡献关系网络。
}

\heading{项目实战}
\entry{2em}{Xp{8em}}{%
    {\heiti 遗传性疾病基因数据的分析和建模}\\
	分析目标:定性与定量相结合的方法找出与遗传性疾病相关的基因位点\\
	主要工作:使用 Pearson 卡方检验降维,岭回归、Lasso 回归和适应性 Lasso 回归做特征提取,利用 Cp、AICC、GCV、BIC 准则做特征选择,logistic分布做损失函数,得到很可能是致病位点的集合。\\
    {\heiti 可重复的定时数据分析报告}\\
	分析目标:基于日志数据运维分析报告\\
	主要工作:在大规模新浪运维日志数据上,基于 Docker、Clickhose 和 R Markdown 制作数据分析报告
}
%%%%%%%%%%%%%%%%%%%%%%%%%%%%%%%%%%%%%%%%%%%%%%%%%%%%%%%%%%%%%%%%%%%%%%%%%%%%%%%

\heading{专业技能}
\entry{2em}{lX}{%
    语言 & R、SQL、\LaTeX 、MATLAB、Python \\
    系统 & Windows、Linux \\
    工具 & RStudio、Github、Markdown、Docker、ggplot2、SparkR \\
    统计 & 数据清洗、探索、处理、建模与可视化、统计计算与绘图 \\
	英语 & 能够流畅地阅读专业领域内的论文和其它文献,并熟练地用英文检索遇到的问题
}

%%%%%%%%%%%%%%%%%%%%%%%%%%%%%%%%%%%%%%%%%%%%%%%%%%%%%%%%%%%%%%%%%%%%%%%%%%%%%%%
%%%%%%%%%%%%%%%%%%%%%%%%%%%%%%%%%%%%%%%%%%%%%%%%%%%%%%%%%%%%%%%%%%%%%%%%%%%%%%%
\heading{社区贡献:统计之都 https://cosx.org/}
\entry{2em}{lX}{%
    R语言做符号计算(一作) {\small\url{https://cosx.org/2016/07/r-symbol-calculate}}\\
    随机数生成及其在统计模拟中的应用(一作) {\small\url{https://cosx.org/2017/05/random-number-generation/}} \\
	漫谈条形图(一作)   {\small\url{https://cosx.org/2017/10/discussion-about-bar-graph/}} \\
	心理学的危机(主审) {\small\url{https://cosx.org/2017/09/psychology-in-crisis/}} \\
	在统计之都的工作:编辑、作者、审稿人、论坛版主\\
	第11届中国R语言会议(北京)演讲嘉宾  2018.05.27\\
}

%%%%%%%%%%%%%%%%%%%%%%%%%%%%%%%%%%%%%%%%%%%%%%%%%%%%%%%%%%%%%%%%%%%%%%%%%%%%%%%
% 如果不需要发表成果,注释这一段即可
% \heading{附:发表成果}
% \vspace{-6em}
% \bibliography{resume}

\end{table}
\end{document}
