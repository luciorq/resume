% !Mode:: "TeX:UTF-8"
% +-----------------------------------------------------------------------------
% | File: resume-zh
% | Author: huxuan
% | E-mail: i(at)huxuan.org
% | Created: 2012-12-18
% | Last modified: 2013-03-16
% | Description:
% |     A Chinese Resume Example in LaTeX based on resumecls
% |
% | Copyrgiht (c) 2012-2013 by huxuan. All rights reserved.
% +-----------------------------------------------------------------------------

\documentclass[color]{resumecls}
\usepackage[default]{sourcesanspro}
\usepackage{sourcecodepro}
% \ctexset{today=small}

\name{黄湘云}
\organization{中国矿业大学(北京)}
\address{北京市海淀区学院路丁11号,100083}
\mobile{+86 188 1097 2907}
\mail{xiangyunfaith@outlook.com}
\homepage{https://xiangyun.netlify.com/}
% \leftfooter{最后更新: \today}
% \rightfooter{\url{http://example.com/resume-zh.pdf}}

\begin{document}

\begin{table}

\maketitle

%%%%%%%%%%%%%%%%%%%%%%%%%%%%%%%%%%%%%%%%%%%%%%%%%%%%%%%%%%%%%%%%%%%%%%%%%%%%%%%
\heading{教育经历}
\entry{2em}{Xrp{8em}}{%
    \heiti{中国矿业大学(北京)} & 北京市 & 2015.09--2018.12 \\
}
\entry{4em}{lXX}{%
    理学硕士学位 & 理学院 & 统计学 \\
	毕业论文:《空间广义线性混合效应模型及其应用》 &  & \\
	研究方向:数据分析与统计计算 & &
}

\entry{2em}{Xrp{8em}}{%
    \heiti{中国矿业大学(北京)} & 北京市 & 2011.09--2015.06 \\
}
\entry{4em}{lXX}{%
    理学学士学位 & 理学院 & 数学与应用数学 \\
	% 毕业论文:《统计方法在区域经济社会发展综合评价中的应用》 &  &  \\
}

%%%%%%%%%%%%%%%%%%%%%%%%%%%%%%%%%%%%%%%%%%%%%%%%%%%%%%%%%%%%%%%%%%%%%%%%%%%%%%
\heading{实习经历}
\entry{2em}{Xp{8em}}{%
    \heiti{北京一流科技有限公司} & 机器学习实习生 \\
}
\entry{4em}{X}{工作的具体内容:}
\entry{6em}{X}{%
    将时域和频域方法与深度学习方法应用于预测电力消费,涉及数据清洗,可视化,指数平滑,
ARMA 和GARCH 模型,贝叶斯时间序列分析工具prophet(包含频谱分析),尖点检测,长短期记忆网络LSTM
及Keras 工具 \\
}

\entry{2em}{Xp{8em}}{%
    \heiti{新浪公司} & 数据分析师实习生 \\
}
\entry{4em}{X}{工作的具体内容:}
\entry{6em}{X}{%
    新浪新闻客户端日志分析,主要使用SQL 从ClickHouse 数据仓库提取数据,R 语言分析和可视化,
并完成日报,用数据分析协助其他部门决策,如服务器资源调度等。\\
}
%%%%%%%%%%%%%%%%%%%%%%%%%%%%%%%%%%%%%%%%%%%%%%%%%%%%%%%%%%%%%%%%%%%%%%%%%%%%%%%
\heading{校园及社会实践}
\entry{2em}{Xp{8em}}{%
    院团委宣传部干事 & 2011.10--2012.06 \\
    校红会学生分会组织部长 & 2012.06--2013.06 \\
	校红十字学生分会组织的爱心支教活动 & 2012.09--2012.10 \\
	大学生科研创新训练项目组长  & 2012.09--2014.12\\
	CDA数据分析师培训      & 2016.04--2016.06 \\
	北京大学高维统计短期课程(高维统计与网络分析) & 2016.06\\
	国家自然科学基金项目(编号:11671398)项目组成员& 2017.01--2020.12\\
	研究生助教(数学建模、数值分析、运筹学等课程) & 2015.09--2016.12
}

%%%%%%%%%%%%%%%%%%%%%%%%%%%%%%%%%%%%%%%%%%%%%%%%%%%%%%%%%%%%%%%%%%%%%%%%%%%%%%%
\heading{资格证书及获奖情况}
\entry{2em}{Xr}{%
    全国大学英语四级(448)和六级(462) & 2011.06--2014.06 \\
    2013年高教社杯全国大学生数学建模竞赛(北京市二等奖) & 2013 \\
	2012年本科生优秀学生奖学金(三等奖学金) & 2012 \\
	2013年本科生优秀学生奖学金(三等奖学金) & 2013 \\
	第29届全国部分地区物理竞赛(三等奖)& 2012\\
	校科技文化节科普创作大赛(优秀奖) & 2013 \\
	校物理实验竞赛(一等奖) & 2013\\
	2015-2016年度研究生优秀学生(一等奖学金) & 2016 \\
	2016-2017年度研究生优秀学生荣誉称号 & 2017\\
}


\end{table}

\begin{table}
%%%%%%%%%%%%%%%%%%%%%%%%%%%%%%%%%%%%%%%%%%%%%%%%%%%%%%%%%%%%%%%%%%%%%%%%%%%%%%
\heading{数据可视化}
\entry{2em}{Xp{8em}}{%
    \heiti{地点} & 起止时间 \\
}
\entry{4em}{X}{实验室名称 \quad 职位}
\entry{6em}{X}{%
    研究方向和具体内容 \\
    发表成果(亦可使用bibtex,像这样\cite{label},见文档最后注释内容) \\
}
%%%%%%%%%%%%%%%%%%%%%%%%%%%%%%%%%%%%%%%%%%%%%%%%%%%%%%%%%%%%%%%%%%%%%%%%%%%%%%%

\heading{专业技能}
\entry{2em}{lX}{%
    语言 & R、SQL、\LaTeX 、MATLAB、Python \\
    系统 & Windows、Linux \\
    软件/工具 & RStudio、Github、Markdown、Docker、ggplot2、SparkR \\
    统计 & 数据清洗、探索、处理、建模与可视化、统计计算与绘图 \\
	英语 & 能够流畅地阅读专业领域内的论文和其它文献,并熟练地用英文检索遇到的问题
}


%%%%%%%%%%%%%%%%%%%%%%%%%%%%%%%%%%%%%%%%%%%%%%%%%%%%%%%%%%%%%%%%%%%%%%%%%%%%%%%
%%%%%%%%%%%%%%%%%%%%%%%%%%%%%%%%%%%%%%%%%%%%%%%%%%%%%%%%%%%%%%%%%%%%%%%%%%%%%%%
\heading{统计之都社区贡献}
\entry{2em}{lX}{%
    R语言做符号计算(一作) & https://cosx.org/2016/07/r-symbol-calculate\\
    随机数生成及其在统计模拟中的应用(一作) & https://cosx.org/2017/05/random-number-generation/ \\
	漫谈条形图(一作) & https://cosx.org/2017/10/discussion-about-bar-graph/ \\
	心理学的危机(主审) & https://cosx.org/2017/09/psychology-in-crisis/ \\
	承担工作 & 主站编辑、作者、审稿人、论坛版主\\
	第11届中国R语言会议(北京)演讲嘉宾 & 2018.05.27\\
}

%%%%%%%%%%%%%%%%%%%%%%%%%%%%%%%%%%%%%%%%%%%%%%%%%%%%%%%%%%%%%%%%%%%%%%%%%%%%%%%
% 如果不需要发表成果,注释这一段即可
\heading{附:发表成果}
\vspace{-6em}
\bibliography{resume}

\end{table}
\end{document}
